\chapter{Research Methodology}

This study employs a \textbf{Mixed-Method Research Design} (Convergent Parallel Design), integrating quantitative analysis of secondary macro-economic data with qualitative insights from primary surveys. This dual approach ensures a holistic understanding of both the ``what'' (Numbers) and the ``why'' (Behavior).

\section{Research Questions}

The study seeks to answer the following research questions:

\begin{enumerate}
    \item \textbf{RQ1:} What is the comparative growth trajectory of UPI transactions versus physical credit card POS transactions in India (2020--2025)?

    \item \textbf{RQ2:} Is there a statistically significant substitution effect between UPI adoption and physical credit card usage?

    \item \textbf{RQ3:} What factors (convenience, rewards, security, speed) drive consumer preference for ``Scan'' (UPI) over ``Swipe'' (Card)?

    \item \textbf{RQ4:} What is the merchant's willingness to accept ``Credit on UPI'' transactions given the associated MDR?

    \item \textbf{RQ5:} Is there a significant difference in the Average Ticket Size (ATS) between physical credit card transactions and UPI-linked credit transactions?
\end{enumerate}

\section{Objectives}

\subsection{Primary Objective}

To evaluate the impact of the adoption of ``RuPay Credit on UPI'' on the transaction volume and growth trajectory of physical credit card swipes.

\subsection{Secondary Objectives}

\begin{enumerate}
    \item \textbf{Trend Analysis:} To analyze the comparative growth rates (CAGR) of UPI QR deployment versus POS Terminal deployment from 2020 to 2025.

    \item \textbf{Consumer Behavior:} To determine the factors (Speed, Rewards, Convenience, Security) driving the shift from ``Swipe'' to ``Scan'' among urban consumers.

    \item \textbf{Merchant Acceptance:} To assess the willingness of Tier-2 and Tier-3 merchants to accept credit payments via QR codes compared to POS machines.

    \item \textbf{Demographic Analysis:} To identify generational differences (Gen Z vs. Millennials vs. Gen X) in payment form factor preferences.

    \item \textbf{Forecasting:} To predict the potential trajectory of physical plastic card issuance over the next five years based on current adoption trends.
\end{enumerate}

\section{Tentative Hypotheses}

Based on the preliminary literature review, the following hypotheses are proposed for statistical testing:

\subsection{Hypothesis 1: The Substitution Effect}

\begin{itemize}
    \item \textbf{$H_{0}$ (Null):} There is no significant negative correlation between the volume of UPI transactions and the volume of physical Credit Card POS transactions.
    \item \textbf{$H_{1}$ (Alternate):} There is a significant negative correlation between the volume of UPI transactions and the volume of physical Credit Card POS transactions (indicating substitution).
\end{itemize}

\subsection{Hypothesis 2: The Ticket Size Differentiation}

\begin{itemize}
    \item \textbf{$H_{0}$ (Null):} There is no significant difference in the Average Ticket Size (ATS) of Physical Credit Card transactions and UPI-linked Credit Card transactions.
    \item \textbf{$H_{1}$ (Alternate):} The Average Ticket Size (ATS) of Physical Credit Card transactions is significantly higher than that of UPI-linked Credit Card transactions.
\end{itemize}

\subsection{Hypothesis 3: Merchant Preference}

\begin{itemize}
    \item \textbf{$H_{0}$ (Null):} Merchants do not show a significant preference for QR codes over POS terminals for accepting credit payments.
    \item \textbf{$H_{1}$ (Alternate):} Merchants show a significant preference for QR codes due to lower perceived setup and maintenance costs.
\end{itemize}

\subsection{Hypothesis 4: Generational Difference}

\begin{itemize}
    \item \textbf{$H_{0}$ (Null):} There is no significant association between age group and preference for UPI over physical cards.
    \item \textbf{$H_{1}$ (Alternate):} Younger age groups (Gen Z) show significantly higher preference for UPI compared to older age groups (Gen X).
\end{itemize}

\section{Research Design}

\begin{itemize}
    \item \textbf{Type:} Descriptive and Exploratory Mixed-Method Design.
    \item \textbf{Duration of Study:} The study covers the fiscal years 2020-21 to 2024-25 to capture the pre-UPI Credit and post-UPI Credit trends.
    \item \textbf{Geographical Scope:}
    \begin{itemize}
        \item \textit{Secondary Data:} Pan-India (National-level RBI/NPCI statistics).
        \item \textit{Primary Data:} Focused on Delhi-NCR (representing a high-adoption urban cluster with diverse demographics).
    \end{itemize}
\end{itemize}

\section{Sampling and Techniques}

\subsection{Target Population}

\begin{enumerate}
    \item \textbf{Consumer Population:} Individuals aged 18--55 residing in Delhi-NCR, possessing both a bank account and a smartphone, and having made at least one digital payment in the past month.

    \item \textbf{Merchant Population:} Retailers in Delhi-NCR currently accepting digital payments (either UPI or Card or both).
\end{enumerate}

\subsection{Sampling Technique}

\textbf{Stratified Random Sampling} is employed to ensure representation across key demographic segments.

\textbf{Consumer Stratification:}
\begin{itemize}
    \item \textbf{Age Groups:} Gen Z (18--25), Millennials (26--40), Gen X (41--55)
    \item \textbf{Income Levels:} Low ($<$₹5 LPA), Middle (₹5--15 LPA), High ($>$₹15 LPA)
\end{itemize}

\textbf{Merchant Stratification:}
\begin{itemize}
    \item \textbf{Business Type:} Unorganized Retail (Kirana/Stalls) vs. Organized Retail (Malls/Chains)
    \item \textbf{Location:} Central Delhi vs. Suburban NCR
\end{itemize}

\subsection{Sample Size}

\begin{itemize}
    \item \textbf{Consumers ($n_1$):} 200 Respondents
    \item \textbf{Merchants ($n_2$):} 50 Respondents
    \item \textbf{Total Sample:} 250 Respondents
\end{itemize}

The sample size is determined using Cochran's formula for sample size calculation, considering a 95\% confidence level and 7\% margin of error, appropriate for a pilot/dissertation-level study.

\section{Source/Instrument of Data}

\subsection{Secondary Data Sources}

Secondary data serves as the backbone for the macro-level trend analysis.

\begin{table}[htbp]
    \centering
    \caption{Secondary Data Sources}
    \vspace{0.3cm}
    \begin{tabular}{|p{3cm}|p{5cm}|p{5cm}|}
    \hline
    \textbf{Source} & \textbf{Publication} & \textbf{Data Points Extracted} \\
    \hline
    Reserve Bank of India (RBI) & Payment System Indicators (Monthly Bulletins) & Credit Card Volume \& Value, POS Terminal Count, Settlement Data \\
    \hline
    National Payments Corporation of India (NPCI) & Product Statistics & UPI Transaction Volume \& Value, QR Code Deployment, RuPay Credit-UPI Stats \\
    \hline
    Industry Reports & PwC, Worldline, NITI Aayog & Market Projections, Competitive Analysis \\
    \hline
    \end{tabular}
    \label{tab:secondary_sources}
\end{table}

\subsection{Primary Data Instruments}

\textbf{Instrument 1: Consumer Questionnaire}

A structured questionnaire using a 5-point Likert Scale (1 = Strongly Disagree, 5 = Strongly Agree) to measure attitudes and preferences.

\textbf{Sample Questions:}
\begin{enumerate}
    \item ``I have stopped carrying my physical wallet because I can pay everywhere with my phone.''
    \item ``I would prefer using a Credit Card on UPI over swiping a plastic card if rewards are equal.''
    \item ``I feel safer scanning a QR code than handing over my card to a waiter/merchant.''
    \item ``I find it inconvenient to locate and remove my physical card for small purchases.''
    \item ``I am aware that my RuPay credit card can be linked to my UPI app.''
\end{enumerate}

\textbf{Instrument 2: Merchant Interview Schedule}

A semi-structured interview schedule with both closed-ended (Likert scale) and open-ended questions.

\textbf{Sample Questions:}
\begin{enumerate}
    \item ``The monthly rental of a POS machine is a significant burden on my business expenses.'' (Likert)
    \item ``I prefer customers to pay via UPI because the settlement is faster than card machines.'' (Likert)
    \item ``If Credit-UPI attracts a 2\% MDR, would you still accept it?'' (Yes/No/Conditional)
    \item ``What would make you consider returning your POS terminal?'' (Open-ended)
\end{enumerate}

\section{Variables}

\subsection{Independent Variables}

\begin{itemize}
    \item UPI Transaction Volume (Macro)
    \item QR Code Deployment Count (Macro)
    \item Age Group of Consumer (Primary)
    \item Income Level of Consumer (Primary)
    \item Type of Merchant (Organized/Unorganized)
\end{itemize}

\subsection{Dependent Variables}

\begin{itemize}
    \item Physical Credit Card Transaction Volume (Macro)
    \item POS Terminal Growth Rate (Macro)
    \item Consumer Preference Score for UPI (Primary)
    \item Merchant Willingness to Accept Credit-UPI (Primary)
\end{itemize}

\subsection{Moderating Variables}

\begin{itemize}
    \item Perceived Security
    \item Reward Structure Availability
    \item Merchant Discount Rate (MDR) Awareness
\end{itemize}

\section{Statistical Tools}

The collected data will be analyzed using \textbf{SPSS (Statistical Package for Social Sciences)} Version 26 and MS Excel.

\subsection{Descriptive Statistics}

\begin{itemize}
    \item \textbf{Mean, Median, Mode:} To summarize central tendencies of demographic and behavioral data.
    \item \textbf{Standard Deviation:} To measure dispersion in responses.
    \item \textbf{Frequency Distribution:} To present demographic profiles.
\end{itemize}

\subsection{Inferential Statistics}

\begin{table}[htbp]
    \centering
    \caption{Statistical Tools and Their Application}
    \vspace{0.3cm}
    \begin{tabular}{|p{4cm}|p{4cm}|p{5cm}|}
    \hline
    \textbf{Statistical Tool} & \textbf{Purpose} & \textbf{Application in Study} \\
    \hline
    Pearson's Correlation Coefficient ($r$) & Measure linear relationship between two variables & Test correlation between UPI volume growth and Credit Card volume growth (H1) \\
    \hline
    Independent Sample t-Test & Compare means of two independent groups & Compare ATS of Physical Card vs. UPI-Credit transactions (H2) \\
    \hline
    Chi-Square Test ($\chi^2$) & Test association between categorical variables & Test if preference for UPI is dependent on age group (H4) \\
    \hline
    ANOVA (Analysis of Variance) & Compare means across multiple groups & Compare preference scores across three age groups \\
    \hline
    CAGR Calculation & Measure compound growth over time & Calculate growth rates of UPI vs. Card transactions \\
    \hline
    Regression Analysis & Predict dependent variable from independent variables & Model factors predicting consumer adoption \\
    \hline
    \end{tabular}
    \label{tab:statistical_tools}
\end{table}

\section{Organization of Data Analysis}

The data analysis will be organized in two distinct phases:

\subsection{Phase 1: Macro-Economic Analysis (Secondary Data)}

\begin{enumerate}
    \item \textbf{Time Series Construction:} Monthly data from RBI/NPCI bulletins (April 2020 -- March 2025) will be compiled into time series datasets.
    \item \textbf{Trend Visualization:} Line graphs and bar charts depicting comparative growth trajectories.
    \item \textbf{CAGR Computation:} Five-year compound annual growth rates for UPI, Credit Cards, POS, and QR codes.
    \item \textbf{Correlation Testing:} Pearson's $r$ between UPI volume and Card volume to test H1.
    \item \textbf{ATS Analysis:} Calculation of Average Ticket Size (Total Value / Total Volume) for each instrument.
\end{enumerate}

\subsection{Phase 2: Behavioral Analysis (Primary Data)}

\begin{enumerate}
    \item \textbf{Data Cleaning:} Removal of incomplete responses, outlier detection.
    \item \textbf{Demographic Profiling:} Frequency tables and pie charts for sample composition.
    \item \textbf{Reliability Testing:} Cronbach's Alpha to test internal consistency of Likert scale items.
    \item \textbf{Hypothesis Testing:} Application of t-tests, Chi-square, and ANOVA as appropriate.
    \item \textbf{Cross-Tabulation:} Analysis of preference patterns across demographic segments.
\end{enumerate}

% --- METHODOLOGY MATRIX TABLE ---
\begin{table}[htbp]
    \centering
    \caption{Methodological Matrix Summary}
    \vspace{0.3cm}
    \begin{tabular}{|p{2.5cm}|p{4cm}|p{6cm}|}
    \hline
    \textbf{Phase} & \textbf{Objective} & \textbf{Tool/Technique} \\
    \hline
    Phase 1 & Analyze Macro Trends (Volume/Value) & Time Series Analysis, CAGR Calculation, Trend Lines (Excel/SPSS) \\
    \hline
    Phase 2 & Test Substitution Hypothesis (H1) & Pearson's Correlation Coefficient ($r$) between UPI and Card Volume \\
    \hline
    Phase 3 & Consumer Preference Analysis & Likert Scale Survey (1-5), Chi-Square Test of Independence, ANOVA \\
    \hline
    Phase 4 & Merchant Viability Analysis & Cost-Benefit Analysis (MDR vs. Device Rental), Descriptive Statistics \\
    \hline
    \end{tabular}
    \label{tab:methodology_matrix}
\end{table}
