\section{Research Methodology}

This study employs a \textbf{Mixed-Method Research Design}, integrating quantitative analysis of secondary macro-economic data with qualitative insights from primary surveys.

\subsection{Research Questions}
\begin{enumerate}
    \item What is the comparative growth trajectory of UPI transactions versus physical credit card POS transactions?
    \item Is there a statistically significant substitution effect between UPI adoption and physical credit card usage?
    \item What factors drive consumer preference for ``Scan'' (UPI) over ``Swipe'' (Card)?
    \item What is the merchant's willingness to accept ``Credit on UPI'' given the associated MDR?
    \item Is there a significant difference in the Average Ticket Size (ATS) between physical and UPI-linked credit transactions?
\end{enumerate}

\subsection{Objectives}
To evaluate the impact of ``RuPay Credit on UPI'' on physical card usage, analyze growth trends, determine consumer drivers, assess merchant acceptance, and forecast future card issuance.

\subsection{Tentative Hypotheses}
\begin{itemize}
    \item \textbf{H1 (Substitution):} Significant negative correlation between UPI volume and physical Credit Card POS volume.
    \item \textbf{H2 (Ticket Size):} ATS of Physical Credit Cards is significantly higher than that of UPI-linked Credit Cards.
    \item \textbf{H3 (Merchant Preference):} Merchants prefer QR codes over POS terminals due to lower costs.
    \item \textbf{H4 (Generational Difference):} Younger age groups show significantly higher preference for UPI.
\end{itemize}

\subsection{Research Design}
Descriptive and Exploratory Mixed-Method Design covering FY 2020-21 to 2024-25. Secondary data is Pan-India; Primary data is from Delhi-NCR.

\subsection{Sampling \& Techniques}
\textbf{Stratified Random Sampling} is used.
\begin{itemize}
    \item \textbf{Consumers:} 200 respondents (stratified by Age and Income).
    \item \textbf{Merchants:} 50 respondents (stratified by Business Type and Location).
    \item \textbf{Total Sample:} 250 respondents.
\end{itemize}

\subsection{Source/Instrument of Data}
\begin{itemize}
    \item \textbf{Secondary Data:} RBI Payment System Indicators, NPCI Product Statistics, Industry Reports.
    \item \textbf{Primary Data:} Structured Consumer Questionnaire (Likert Scale) and Semi-structured Merchant Interview Schedule.
\end{itemize}

\subsection{Variables}
\begin{itemize}
    \item \textbf{Independent:} UPI Volume, QR Deployment, Age, Income, Merchant Type.
    \item \textbf{Dependent:} Credit Card Volume, POS Growth, Consumer Preference, Merchant Willingness.
    \item \textbf{Moderating:} Security, Rewards, MDR Awareness.
\end{itemize}

\subsection{Statistical Tools}
Data will be analyzed using SPSS and Excel. Tools include Descriptive Statistics (Mean, SD), Correlation Analysis (Pearson's r), t-Tests, Chi-Square Tests, ANOVA, and CAGR calculation.

\subsection{Organization of Data Analysis}
\begin{itemize}
    \item \textbf{Phase 1 (Macro):} Trend visualization, CAGR computation, and Correlation testing of secondary data.
    \item \textbf{Phase 2 (Behavioral):} Demographic profiling, Hypothesis testing, and Cross-tabulation of primary survey data.
\end{itemize}
