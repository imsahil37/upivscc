\section{Rationale of the Study}

The introduction of \textbf{``Credit on UPI''}---specifically the RBI's approval for RuPay Credit Cards to be linked to UPI apps in June 2022---addresses the Infrastructure Asymmetry directly. This policy shift allows a user to scan a standard UPI QR code but choose their credit card as the source of funds, rather than their bank account.

\subsubsection{Rationale for Merchants: Zero CAPEX and Revenue Expansion}

For the 65 million+ merchants accepting UPI, this convergence is transformative.

\begin{itemize}
    \item \textbf{Zero CAPEX:} They do not need to upgrade to a POS machine to accept credit customers. The existing QR sticker works.
    \item \textbf{Higher Ticket Sizes:} Credit card users typically have higher disposable incomes and spend more per transaction, potentially increasing average bill values.
    \item \textbf{MDR Relief:} For small merchants (transactions under \rupee 2,000), the MDR on RuPay UPI credit transactions is often subsidized or waived by government intervention.
\end{itemize}

\subsubsection{Rationale for Issuers: Distribution Velocity}

For banks, the ``Credit on UPI'' model solves the logistics of customer acquisition and activation.

\begin{itemize}
    \item \textbf{Virtual Issuance:} Banks can issue virtual RuPay credit cards instantly via mobile apps, eliminating the need for physical manufacturing, courier logistics, and activation calls. In 2024, \textbf{50\%} of new credit cards issued were virtual.
    \item \textbf{Transaction Frequency:} The ease of scanning a QR code for small purchases results in higher engagement. UPI-enabled credit cards register an average of \textbf{40 transactions per month}, which is eight times higher than the 5 transactions typical of physical cards.
    \item \textbf{Data Richness:} Every UPI transaction generates rich merchant data (location, category, time), enabling better credit risk modeling.
\end{itemize}

\subsubsection{Rationale for Consumers: Convenience and Access}

\begin{itemize}
    \item \textbf{Ubiquitous Acceptance:} Credit can now be used at street vendors, auto-rickshaws, and small shops that never had POS machines.
    \item \textbf{Simplified Experience:} One app, one scan, multiple funding sources (bank account or credit line).
    \item \textbf{Rewards Democratization:} Credit card rewards, previously limited to organized retail, can now be earned on everyday micro-purchases.
\end{itemize}

\subsubsection{Academic and Policy Rationale}

This study is timely and necessary because:

\begin{enumerate}
    \item \textbf{Policy Evaluation:} The RBI's 2022 circular is now mature enough (3 years) for preliminary impact assessment.
    \item \textbf{Industry Strategy:} Banks, fintechs, and card networks need empirical data to inform strategic decisions on physical vs. virtual issuance.
    \item \textbf{Academic Contribution:} This study fills the identified gap in ``intra-digital'' payment competition research.
\end{enumerate}
