\section{Research Gap}

Despite the extensive literature on digital payments in India, a critical void exists in the academic discourse.

\subsubsection{The ``Digital vs. Digital'' Blind Spot}

The majority of existing studies (2016--2022) focus on the binary competition between \textbf{Cash vs. Digital}.

\begin{itemize}
    \item \textbf{Existing Focus:} How UPI is killing cash.
    \item \textbf{Missing Focus:} How UPI is cannibalizing other digital instruments (Credit/Debit Cards).
\end{itemize}

There is insufficient longitudinal analysis on the \textit{intra-digital} cannibalization. As cash recedes to the margins, the real battle is now between the QR Code and the Plastic Card. Current literature fails to adequately address this ``Form Factor War'' within the digital payment ecosystem itself.

\subsubsection{The ``Credit on UPI'' Novelty}

The specific policy intervention allowing RuPay Credit Cards on UPI is nascent (RBI circular issued June 2022, effectively operational at scale only from late 2023). Consequently, there is almost no academic literature analyzing:

\begin{enumerate}
    \item The shift in Average Ticket Size (ATS) when a user switches from a Physical Credit Card to a UPI-linked Credit Card.
    \item The merchant's reaction to ``Credit UPI'' transactions which carry an MDR (unlike standard UPI), and whether this will lead to resistance or surcharge practices.
    \item The generational and demographic drivers of adoption across different age cohorts.
    \item The infrastructure implications for the POS terminal industry as QR acceptance expands.
\end{enumerate}

\subsubsection{Methodological Gap}

Most existing studies rely heavily on either:
\begin{itemize}
    \item \textbf{Aggregate Secondary Data:} Macro-level RBI statistics without consumer-level behavioral insights.
    \item \textbf{Qualitative Case Studies:} Anecdotal evidence without statistical rigor.
\end{itemize}

There is a need for \textbf{mixed-method research} that combines quantitative trend analysis with primary behavioral data to understand both the ``what'' (macro trends) and the ``why'' (consumer motivations).

\vspace{1cm}

\noindent\fbox{\parbox{\textwidth}{
\textbf{Gap Statement:} This dissertation aims to bridge this specific gap by analyzing post-2023 data on ``Credit on UPI'' adoption, combining RBI/NPCI secondary statistics with primary survey data from consumers and merchants to understand the behavioral and economic dynamics of the ``Death of Plastic'' phenomenon.
}}
