\subsubsection{Scope and Limitations}

\subsubsection{Scope of the Study}

\subsubsection{Geographical Scope}

\begin{itemize}
    \item \textbf{Secondary Data:} Pan-India analysis based on national RBI and NPCI statistics.
    \item \textbf{Primary Data:} Restricted to the National Capital Region (NCR), including Delhi, Gurgaon, Noida, and Faridabad.
\end{itemize}

\subsubsection{Temporal Scope}

\begin{itemize}
    \item The study covers the period from \textbf{FY 2020-21 to FY 2024-25} (5 fiscal years).
    \item This period captures the pre-Credit on UPI era (2020--2022) and the post-Credit on UPI era (2023--2025).
\end{itemize}

\subsubsection{Subject Matter Scope}

\begin{itemize}
    \item The study focuses specifically on \textbf{retail point-of-sale (POS)} transactions.
    \item \textbf{Excludes:} B2B (Business to Business) transfers, Net Banking transactions, ATM withdrawals, and e-commerce/online card-not-present transactions.
    \item \textbf{Includes:} Physical card swipes at POS terminals, UPI QR code scans, and Credit-linked UPI transactions.
\end{itemize}

\subsubsection{Demographic Scope}

\begin{itemize}
    \item Consumer respondents: Ages 18--55 years.
    \item Merchant respondents: Retail businesses (both organized and unorganized).
\end{itemize}

\subsubsection{Limitations of the Study}

\begin{enumerate}
    \item \textbf{Data Nascent Stage:} ``Credit on UPI'' is a relatively new feature (gaining traction only from late 2023). Long-term longitudinal data is not yet available. The study relies on early-trend extrapolation, which may not fully capture mature adoption patterns.

    \item \textbf{Self-Reporting Bias:} Survey respondents may overestimate their digital usage due to ``social desirability bias'' (desire to appear tech-savvy). This is a common limitation in behavioral research.

    \item \textbf{Regional Bias:} Urban respondents in Delhi-NCR may have higher connectivity, smartphone penetration, and digital literacy than the national average. The findings may overestimate adoption rates compared to Tier-3/rural India.

    \item \textbf{RuPay Exclusivity:} As of the study period, ``Credit on UPI'' is primarily available only for RuPay credit cards. Visa and Mastercard cardholders (who represent a larger market share) cannot participate, limiting the generalizability of adoption trends.

    \item \textbf{MDR Policy Uncertainty:} Government policies on MDR subsidies for small merchants are subject to change. Any policy shift could significantly alter merchant acceptance patterns, affecting the validity of long-term projections.

    \item \textbf{Sample Size Constraints:} The sample size of 250 respondents (200 consumers + 50 merchants) is adequate for a dissertation-level study but may not capture the full heterogeneity of the Indian market. Larger-scale studies would be needed for policy-level generalizations.

    \item \textbf{Exclusion of E-commerce:} The study focuses on offline/in-store transactions. The dynamics of online card-not-present (CNP) transactions, where physical cards compete with saved card details rather than QR codes, are not addressed.

    \item \textbf{Rapidly Evolving Landscape:} The digital payments ecosystem in India is evolving rapidly. New features (e.g., Visa/Mastercard on UPI, UPI Lite, UPI 123Pay) may emerge during or after the study period, potentially altering the competitive dynamics analyzed herein.

    \item \textbf{Merchant Honesty:} Merchants may underreport their true MDR avoidance strategies (such as asking customers to pay extra or refusing card payments) due to fear of regulatory scrutiny.
\end{enumerate}

\subsubsection{Delimitations}

To maintain focus and feasibility, the following delimitations are consciously applied:

\begin{itemize}
    \item The study does not attempt to analyze the technical architecture of UPI or card networks in depth; it focuses on adoption and behavioral outcomes.
    \item International comparisons (Brazil, China) are used for contextual framing only; no primary data is collected from foreign markets.
    \item The study does not evaluate the profitability or financial performance of individual banks or payment companies.
\end{itemize}
