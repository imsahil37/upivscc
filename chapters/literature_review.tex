\section{Literature Review}

The literature on digital payments reveals a shift from card-based systems to mobile-first ecosystems. This review synthesizes key theories and empirical studies relevant to the study.

\subsection{Theoretical Framework}
Scholars like Evans and Schmalensee (2005) established the foundational understanding of two-sided markets in payment networks. They argued that the value of a network is proportional to the number of users on both sides (merchants and consumers). In this context, UPI's open architecture has allowed it to scale much faster than closed-loop card networks.

\subsection{UPI vs. Cards}
Research by Bhattacharya (2022) highlights that UPI's success lies in its zero-MDR policy for P2P and low-value P2M transactions, which removed the barrier to entry for small merchants. In contrast, Agarwal (2023) notes that credit cards have remained an elite product due to high issuance costs and strict underwriting. The cost of POS terminals (CAPEX) and MDR (OPEX) are cited as the primary reasons for the low penetration of credit cards in India.

\subsection{The ``Credit on UPI'' Disruption}
Early studies on this specific feature (NITI Aayog, 2023) suggest it solves the ``last mile'' problem of credit availability. However, potential conflicts regarding MDR distribution between the issuer, acquirer, and network provider remain a key area of debate (PwC, 2023). Industry reports suggest that while consumers prefer the convenience of scanning a QR code, the economic model for issuers is under pressure due to lower MDR on UPI transactions compared to physical card swipes.

\subsection{Consumer Behavior and Technology Adoption}
Studies on technology adoption (Davis, 1989) suggest that Perceived Ease of Use and Perceived Usefulness are primary drivers. For Gen Z, the mobile phone is the primary interface for all interactions, making the physical card an inconvenient artifact (Deloitte, 2024). Security concerns, however, remain a moderating variable, with some older demographics still perceiving physical cards as more secure than mobile wallets.
