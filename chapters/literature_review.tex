\section{Literature Review}

This section reviews theoretical frameworks and empirical studies relevant to the shift from physical cards to digital interfaces.

\subsection{Theoretical Frameworks}

\textbf{Technology Acceptance Model (TAM):} Davis's (1989) TAM suggests adoption is driven by Perceived Usefulness (PU) and Perceived Ease of Use (PEOU).
\begin{itemize}
    \item \textbf{PU:} ``Credit on UPI'' offers liquidity for micro-transactions, a utility previously unavailable.
    \item \textbf{PEOU:} The ``Scan and Pay'' workflow reduces the friction of physical cards (locating wallet, inserting card, entering PIN).
\end{itemize}

\textbf{Unified Theory of Acceptance and Use of Technology (UTAUT):} Venkatesh et al. (2003) identify Performance Expectancy, Effort Expectancy, Social Influence, and Facilitating Conditions.
\begin{itemize}
    \item \textbf{Social Influence:} The visual dominance of QR codes and peer adoption drives usage.
    \item \textbf{Facilitating Conditions:} High smartphone penetration and the RBI's regulatory support enable this shift.
\end{itemize}

\subsection{Empirical Studies}

\textbf{Global Parallels:} Studies on Brazil's \textbf{Pix} system show a strong substitution effect between instant payments and cash, with a ``soft substitution'' for debit cards (Duarte et al., 2022). In China, the ``Super App'' ecosystem (Alipay/WeChat) has shown that integrated platforms offer superior user engagement compared to standalone cards.

\textbf{The Indian Context:} NITI Aayog (2023) highlights a volume vs. value divergence: UPI dominates volume, while credit cards retain value share. However, RBI surveys indicate that MDR remains a critical barrier for small merchants, driving them towards UPI over POS terminals. PwC projections suggest ``Credit on UPI'' could capture 15--20\% of credit card transaction value by 2029.
