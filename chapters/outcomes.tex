\section{Expected Outcomes}

Based on the preliminary literature review and the research design, the following outcomes are anticipated from this study:

\subsubsection{Expected Findings}

\subsubsection{1. Validation of the ``Partial Substitution'' Hypothesis}

The study expects to find evidence supporting the Alternate Hypothesis ($H_1$) with a nuance:

\begin{itemize}
    \item \textbf{Expected Finding:} There will be a statistically significant negative correlation between UPI growth and \textit{Debit Card} POS usage (strong substitution).
    \item \textbf{Nuance for Credit Cards:} For Credit Cards, the substitution will be ``partial''---significant for low-ticket transactions ($<$\rupee 2,000) but not for high-ticket transactions ($>$\rupee 5,000).
    \item \textbf{Implication:} A ``Bifurcated Spending Model'' where UPI captures the ``Convenience Economy'' and plastic cards retain the ``Rewards Economy.''
\end{itemize}

\subsubsection{2. Confirmation of the ATS Gap}

The study expects to confirm that:

\begin{itemize}
    \item Physical Credit Card ATS will be significantly higher ($\sim$\rupee 5,000--\rupee 5,500) than UPI ATS ($\sim$\rupee 1,400--\rupee 1,500).
    \item UPI-linked Credit Card ATS will fall in between ($\sim$\rupee 2,500--\rupee 3,500), capturing the mid-ticket segment.
\end{itemize}

\subsubsection{3. Generational Divergence}

Expected demographic patterns:

\begin{itemize}
    \item \textbf{Gen Z (18--25):} Overwhelming preference for UPI ($>$75\%), ``wallet-less'' behavior.
    \item \textbf{Millennials (26--40):} Mixed preference, driven by reward optimization.
    \item \textbf{Gen X (41+):} Continued reliance on physical cards due to trust and habit.
\end{itemize}

\subsubsection{4. Merchant MDR Resistance}

Expected merchant behavior:

\begin{itemize}
    \item $>$85\% of small/unorganized merchants will resist MDR on Credit-UPI.
    \item $>$60\% of merchants will consider returning POS terminals if Credit-UPI adoption grows.
\end{itemize}

\subsubsection{Contribution to Knowledge}

\begin{enumerate}
    \item \textbf{Academic:} First comprehensive study on ``intra-digital'' payment cannibalization in the Indian context, bridging the identified research gap.

    \item \textbf{Managerial:} Actionable insights for banks on virtual-first issuance strategies and reward restructuring.

    \item \textbf{Policy:} Evidence-based recommendations for RBI on MDR rationalization to balance merchant welfare and credit ecosystem sustainability.

    \item \textbf{Industry:} Data-driven forecast for the POS terminal industry regarding potential obsolescence risks.
\end{enumerate}

\subsubsection{Practical Recommendations (Expected)}

\subsubsection{For Banks and Card Issuers}

\begin{enumerate}
    \item Adopt a ``Virtual-First'' Issuance Strategy---issue physical cards only on request.
    \item Revamp reward structures to incentivize high-frequency UPI scans, not just high-value swipes.
\end{enumerate}

\subsubsection{For the Reserve Bank of India}

\begin{enumerate}
    \item Implement tiered MDR for Credit-UPI (e.g., 0.5\% for $<$\rupee 2,000; 1.5\% cap for higher).
    \item Launch consumer education campaigns on Credit-UPI security features.
\end{enumerate}

\subsubsection{For Fintechs and Payment Apps}

\begin{enumerate}
    \item Build better UI/UX for credit management within UPI apps.
    \item Display ``Available Credit Limit'' prominently on scan screens.
\end{enumerate}
