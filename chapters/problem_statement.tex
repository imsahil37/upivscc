\chapter{Statement of the Problem}

The Indian financial ecosystem is currently witnessing a ``Form Factor War.'' The legacy infrastructure (POS Terminals) is capital-intensive and geographically restricted to urban clusters. The challenger infrastructure (QR Codes) is asset-light and ubiquitous.

\section{The Infrastructure Asymmetry}

This study defines the core phenomenon as \textbf{``Infrastructure Asymmetry''}---a widening gap between the capability to \textit{issue} credit instruments and the capability to \textit{accept} them at the point of sale.

\subsection{The Physical Ceiling: Limitations of POS Terminal Deployment}

The legacy credit card ecosystem relies entirely on the Point-of-Sale (POS) terminal. These terminals are expensive to deploy and maintain.

\begin{itemize}
    \item \textbf{POS Statistics:} As of late 2024/2025, the total number of deployed POS terminals in India stands at approximately \textbf{8.9 million to 11 million}.
    \item \textbf{Growth Rate:} The growth of POS terminals is steady but slow, averaging around 17\% to 23\% year-on-year.
    \item \textbf{Cost Structure:} Each terminal requires CAPEX of \rupee 8,000--\rupee 12,000, plus monthly rental (\rupee 300--\rupee 500) and connectivity charges.
\end{itemize}

\subsection{The Digital Explosion: The Proliferation of QR Codes}

In stark contrast, the UPI acceptance infrastructure---built on asset-light Quick Response (QR) codes---has exploded exponentially.

\begin{itemize}
    \item \textbf{QR Code Statistics:} The number of UPI QR codes deployed across India has surged to between \textbf{633 million and 657 million}.
    \item \textbf{Growth Rate:} The deployment of QR codes is witnessing hyper-growth, with year-on-year increases ranging from \textbf{91.5\% to 126\%}.
    \item \textbf{Cost Structure:} A QR code is essentially free---a printed sticker with zero maintenance cost.
\end{itemize}

\subsection{The Asymmetry Gap: 11 Million vs. 650 Million}

This divergence creates a massive acceptance gap. There are roughly \textbf{60 times more QR codes than POS terminals} in India.

\begin{itemize}
    \item \textbf{The ``Credit Class'' Exclusion:} Until the introduction of Credit on UPI, a consumer holding a credit card could only use it at the $\sim$11 million locations with POS terminals.
    \item \textbf{The ``Debit Class'' Ubiquity:} A consumer using UPI could pay at over 650 million locations.
\end{itemize}

\section{The Core Problem}

Banks and Payment Networks (Visa/Mastercard) have invested billions of dollars over decades in building the card issuance and acceptance network. If the physical card becomes obsolete, this entire infrastructure risks becoming a ``stranded asset.'' Furthermore, the economic model of credit cards relies on MDR. If UPI commoditizes credit payments, the profitability of the credit ecosystem faces an existential threat.

\vspace{0.5cm}

\noindent\fbox{\parbox{\textwidth}{
\textbf{Central Research Question:} \textit{Is the convenience of the QR code powerful enough to dismantle the established habit of the card swipe, and if so, what are the economic implications for the banking sector and payment infrastructure providers?}
}}
