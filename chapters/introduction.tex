\section{Introduction (Background of the Study)}

The history of money is fundamentally a history of reducing friction. Throughout human civilization, economic exchange has evolved from the cumbersome trade of physical commodities to the seamless transfer of digital information. This trajectory has been defined by a relentless drive to decouple ``value'' from ``matter.'' To understand the current disruption---where the plastic credit card is being displaced by the virtual QR code---it is essential to trace the lineage of payment instruments, specifically the transition from tangible currency to the plastic era.

\subsection{Evolution of Money: From Barter to Plastic}
The earliest economic systems relied on barter, which was inefficient due to the need for a ``double coincidence of wants.'' Commodity money (e.g., cattle, grain) and later metallic currency resolved some of these issues but were heavy and risky to transport. Paper fiat currency marked a significant shift, separating value from material worth, but still required physical exchange.

The ``Plastic Era'' began in 1950 with the Diners Club, the first universal charge card. This concept evolved with the introduction of PVC cards by Visa and Mastercard, which became the global standard for credit. However, the physical properties of PVC are now becoming a limitation in a digital-first economy.

\subsection{The Dominance of Card Networks}
The global digital commerce system has long been underpinned by the Four-Party Model involving the Cardholder, Merchant, Acquirer, and Issuer. While robust, this model suffers from high costs, particularly the Merchant Discount Rate (MDR), and significant capital expenditure for Point-of-Sale (POS) terminals. This has limited credit acceptance in developing economies like India to affluent urban centers.

\subsection{The Indian Digital Payment Revolution}
India disrupted this status quo with the launch of the Unified Payments Interface (UPI) in 2016. UPI leverages a Virtual Payment Address (VPA) for real-time settlement, unlike the card networks' T+1 cycle. UPI's growth has been exponential, accounting for nearly 80\% of digital payments in India by FY 2023-24.

\subsection{The Concept of ``Credit on UPI''}
The RBI's 2022 decision to allow RuPay Credit Cards on UPI links the credit instrument to the ubiquitous QR code infrastructure. This innovation threatens to make the physical credit card obsolete, as users can now access credit at millions of small merchants who do not own POS machines.
