\section{Introduction (Background of the Study)}

The history of money is fundamentally a history of reducing friction. The transition from tangible currency to the ``Plastic Era'' began with the Diners Club in 1950, introducing the concept of a universal charge card. This evolved into the PVC cards we use today, standardized by Visa and Mastercard. However, the physical properties of PVC are now becoming a limitation in a digital-first economy.

\subsection{The Dominance of Card Networks}

The global digital commerce system has long been underpinned by the Four-Party Model involving the Cardholder, Merchant, Acquirer, and Issuer, connected by a Network. While robust, this model suffers from high costs, particularly the Merchant Discount Rate (MDR), and significant capital expenditure for Point-of-Sale (POS) terminals. This has limited credit acceptance in developing economies like India to affluent urban centers.

\subsection{The Indian Digital Payment Revolution}

India disrupted this status quo with the launch of the Unified Payments Interface (UPI) in 2016. UPI leverages a Virtual Payment Address (VPA) for real-time settlement, unlike the card networks' T+1 cycle. UPI's growth has been exponential, accounting for nearly 80\% of digital payments in India by FY 2023-24.

\subsection{The Concept of ``Credit on UPI''}

The RBI's 2022 decision to allow RuPay Credit Cards on UPI links the credit instrument to the ubiquitous QR code infrastructure. This innovation threatens to make the physical credit card obsolete, as users can now access credit at millions of small merchants who do not own POS machines.
