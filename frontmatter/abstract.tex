\nonumchapter{Abstract}

The Indian financial ecosystem is witnessing an unprecedented ``Form Factor War'' between the traditional plastic credit card and the emerging QR-code-based ``Credit on UPI'' payment mechanism. While the Unified Payments Interface (UPI) has revolutionized bank-to-bank transfers since its launch in 2016, the Reserve Bank of India's 2022 policy decision to allow RuPay Credit Cards to be linked to UPI apps has created a direct challenge to the dominance of the global card networks (Visa and Mastercard) and the physical plastic instrument itself.

This study investigates whether the convenience of the QR code is powerful enough to dismantle the established habit of the card swipe. The research employs a mixed-method approach, combining quantitative analysis of secondary data from RBI Payment System Indicators (2020--2025) with primary survey data collected from 200 consumers and 50 merchants in the Delhi-NCR region.

The analysis highlights a significant ``Infrastructure Asymmetry,'' with India having around 11 million POS terminals but over 650 million UPI QR codes, resulting in a 60:1 acceptance gap. It also identifies a ``Bifurcated Spending Model,'' where consumers use UPI for transactions under ₹2,000 and retain credit cards for those above ₹5,000. The Average Ticket Size (ATS) gap has widened, with credit cards at ₹5,300 and UPI at ₹1,480, marking a ratio of 3.58:1.

Primary data indicates significant generational divergence: 78\% of Gen Z respondents (18--25 years) frequently leave home without a physical wallet, compared to only 12\% of Gen X respondents. Furthermore, 90\% of small merchants resist the Merchant Discount Rate (MDR) associated with Credit-UPI transactions, posing a critical barrier to adoption.

The study concludes that the dominance of plastic cards as payment methods is waning, giving way to an ``Invisible Credit'' model where the funding source is separate from the physical card. It suggests that banks adopt ``virtual-first'' issuance strategies and that regulators implement tiered MDR structures for a smoother transition.
